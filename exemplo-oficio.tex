% Exemplo de ofício do CACo -- Universidade Federal do Piauí - UFPI
% 2012-12-15 by alemedeiros
\documentclass[12pt]{article}
\usepackage[brazilian]{babel}
\usepackage[utf8]{inputenc}

\usepackage{oficio-caco}

% Insira as informações da gestão, nome e cargo de quem vai assinar o ofício.
\gestao{Acadêmico do Curso de Bacharelado em Sistemas de Informação}
\nome{Esdras Fragoso da Silva Neto}
\cargo{Acadêmico da Universidade Federal do Piauí – UFPI }

% Número e título do ofício.
\oficio{001/2017}
\titulo{ Senhora Secretaria de Educação Maria Dias Cavalcante Vieira.}

\begin{document}

% Aqui vem apenas o texto do conteúdo do ofício.
Venho por meio deste documento rogar a vossa senhoria a autorização para implantação de um Gerenciador Pedagógico Informatizado  em duas ou mais escolas da rede publica municipal, sendo estas divididas em dois grupos  um para implantar o sistema  e outro para servir de métrica para a curva de aprendizado dos estudantes expostos a ferramenta.  Com o objetivo de realizar experimento cientifico para aferir a eficiência do software.  Para que esta meta seja alcançada deve-se realizar avaliações diagnosticas para verificar o nível de aprendizado  dos estudantes nas fases de pré e pós-implantação do sistema, comparando os resultados obtidos por alunos expostos a aplicação com os de alunos que não foram expostos, obtendo os dados estatísticos necessários para o desenvolvimento da monografia. 

Sugiro a E.E.F. Jerônimo Alves Bezerra para Implantação do software, pois já tenho  familiaridade com todos os profissionais que lá trabalham, por isto seria fácil repassar para os mesmos as funcionalidades do sistema fazendo com que o desempenho do mesmo não seja comprometido por inaptidão em sua utilização. 

Na Certeza de vosso pronto Atendimento, apresento o meu préstimo de estima e consideração. Atenciosamente:

\end{document}
